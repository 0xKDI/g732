\section{Тaблицы}

Простая таблица с номером~\ref{tab:simple}.

% \renewcommand{\arraystretch}{1.3}
\begin{table}[H]
\caption{Простая таблица}\label{tab:simple}
\begin{tabular}{|c|c|c|}
    \hline
    11 & 12 & 13\\\hline
    21 & 22 & 23\\\hline
    31 & 32 & 33\\\hline
\end{tabular}
\end{table}

\begin{table}[H]
\caption{Tabularx}\label{tab:x}
\begin{tabularx}{0.8\textwidth} { 
  | >{\raggedright\arraybackslash}X 
  | >{\centering\arraybackslash}X 
  | >{\raggedleft\arraybackslash}X | }
 \hline
     item 11 & item 12 & item 13\\\hline
     item 21  & item 22  & item 23\\\hline
\end{tabularx}
\end{table}

\begin{table}[H]
    \caption{Сложная таблица}\label{tab:hd}
\begin{tabular}{|*{10}{c|}}
    \hline
    \multirow{2}{*}{Параметр \(x_i\)} &
    \multicolumn{4}{c|}{Параметр \(x_j\)} &
    \multicolumn{2}{c|}{Первый шаг} &
    \multicolumn{2}{c|}{Второй шаг}\\\cline{2-9} &
    \(X_1\) & \(X_2\) & \(X_3\) & \(X_4\) & 
    \(w_i\) & \(K_{\text{в}i}\) &
    \(w_i\) & \(K_{\text{в}i}\)\\\hline
    \(X_1\) & 1 & 1 & 1.5 & 1.5 & 5 & 0.31 & 19 & 0.32\\\hline
    \(X_2\) & 1 & 1 & 1.5 & 1.5 & 5 & 0.31 & 19 & 0.32\\\hline
    \(X_3\) & 1 & 1 & 1.5 & 1.5 & 5 & 0.31 & 19 & 0.32\\\hline
    \(X_4\) & 1 & 1 & 1.5 & 1.5 & 5 & 0.31 & 19 & 0.32\\\hline
    \multicolumn{5}{|c|}{Итого:} & 16 & 1 & 59.5 & 1\\\hline
\end{tabular}
\end{table}

\begin{longtable}{|l|l|}
    \caption{Длинная таблица}\label{tab:long}\\\hline
    \multicolumn{1}{|c|}{\textbf{test}} & 
    \multicolumn{1}{c|}{\textbf{test2}} \\\hline 
\endfirsthead
    \caption*{Продолжение таблицы \ref{tab:long}}
\endhead
     Lots of lines & like this\\\hline
     Lots of lines & like this\\\hline
     Lots of lines & like this\\\hline
     Lots of lines & like this\\\hline
     Lots of lines & like this\\\hline
     Lots of lines & like this\\\hline
     Lots of lines & like this\\\hline
     Lots of lines & like this\\\hline
     Lots of lines & like this\\\hline
     Lots of lines & like this\\\hline
     Lots of lines & like this\\\hline
     Lots of lines & like this fffffffffffffffffffffffffffffffffffffffffffffffffffffffffffffffffffffffffffffffffffffffffffffffffffffff\\\hline
\end{longtable}

\begin{table}[H]
    \caption{Tabulary}\label{tab:y}
  \begin{tabulary}{\textwidth}{|L|L|L|}
  \hline
    Short sentences & \# & Long sentences \\\hline
    This is short. & 173 & This is much loooooooonger, because there are many more words. \\
    \hline
  \end{tabulary}
\end{table}

В таблице~\ref{tab:xltabular} представлен xltabular, на данный момент
самая юзабельная таблица, включает в себя преимущества xtabular (автоматическая ширина
таблицы)
и longtable (автоматический перенос со всеми вытекающими).

Если компилятор ругается на overfull при использовании xltabular, то это значит, что
контент в таблице вылезает за пределы страницы (страницы, а не текста), с другими таблицами
в таком случае контент просто вылезет за пределы страницы (таблица~\ref{tab:long} ласт строка).

Еще добавил alias'ы, а то много места команты занимают:

\textbackslash{}mc=\textbackslash{}multicolumn

\textbackslash{}mr=\textbackslash{}multirow

\begin{xltabular}{\textwidth}{|l|l|l|l|}
  \caption{xltabular}\label{tab:xltabular}\\\hline
  \thead{Типичное название\\ столбца} &
  \thead{Ещё\\ столбец} &
  \thead{И ещё один} &
  \mc[l]{Можно использовать\\ \textbackslash{}mc вместо \textbackslash{}thead } \\\hline
  \endfirsthead
  \caption*{Продолжение таблицы \ref{tab:xltabular}}
  \endhead
  \endfoot
  \endlastfoot
  типичная ячейка                   & 2020 & ещё ячейка с чем-то        & и ещё одна \\\hline
  \mc[l]{много\\ текста для ячейки} & 2020 & ещё ячейка                 & и ещё одна \\\hline
  типичная ячейка                   & 2020 & \mc[r]{а это ячейка такая} & и ещё одна \\\hline
  типичная ячейка                   & 2020 & \mc[c]{ещё ячейка}         & и ещё одна \\\hline
\end{xltabular}
